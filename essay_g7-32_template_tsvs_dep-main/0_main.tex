\documentclass[a4paper,14pt,oneside,openany]{memoir}

%%% Задаем поля, отступы и межстрочный интервал %%%

\usepackage[left=30mm, right=15mm, top=20mm, bottom=20mm]{geometry} % Пакет geometry с аргументами для определения полей
\pagestyle{plain} % Убираем стандарные для данного класса верхние колонтитулы с заголовком текущей главы, оставляем только номер страницы снизу по центру
\parindent=1.25cm % Абзацный отступ 1.25 см, приблизительно равно пяти знакам, как по ГОСТ
\usepackage{indentfirst} % Добавляем отступ к первому абзацу
%\linespread{1.3} % Межстрочный интервал (наиболее близко к вордовскому полуторному) - тут вместо этого используется команда OnehalfSpacing*

%%% Задаем языковые параметры и шрифт %%%

\usepackage[english, russian]{babel}                % Настройки для русского языка как основного в тексте
\babelfont[russian]{rm}{Times New Roman}                     % TMR в качестве базового roman-щрифта



%%% Задаем стиль заголовков и подзаголовков в тексте %%%

\setsecnumdepth{subsection} % Номера разделов считать до третьего уровня включительно, т.е. нумеруются только главы, секции, подсекции
\renewcommand*{\chapterheadstart}{} % Переопределяем команду, задающую отступ над заголовком, чтобы отступа не было
\renewcommand*{\printchaptername}{} % Переопределяем команду, печатающую слово "Глава", чтобы оно не печалось
%\renewcommand*{\printchapternum}{} % То же самое для номера главы - тут не надо, номер главы оставляем
\renewcommand*{\chapnumfont}{\normalfont\bfseries} % Меняем стиль шрифта для номера главы: нормальный размер, полужирный
\renewcommand*{\afterchapternum}{\hspace{1em}} % Меняем разделитель между номером главы и названием
\renewcommand*{\printchaptertitle}{\normalfont\bfseries\centering\MakeUppercase} % Меняем стиль написания для заголовка главы: нормальный размер, полужирный, центрированный, заглавными буквами
\setbeforesecskip{20pt} % Задаем отступ перед заголовком секции
\setaftersecskip{20pt} % Ставим такой же отступ после заголовка секции
\setsecheadstyle{\raggedright\normalfont\bfseries} % Меняем стиль написания для заголовка секции: выравнивание по правому краю без переносов, нормальный размер, полужирный
\setbeforesubsecskip{20pt} % Задаем отступ перед заголовком подсекции
\setaftersubsecskip{20pt} % Ставим такой же отступ после заголовка подсекции
\setsubsecheadstyle{\raggedright\normalfont\bfseries}  % Меняем стиль написания для заголовка подсекции: выравнивание по правому краю без переносов, нормальный размер, полужирный

%%% Задаем параметры оглавления %%%

\addto\captionsrussian{\renewcommand\contentsname{Содержание}} % Меняем слово "Оглавление" на "Содержание"
\setrmarg{2.55em plus1fil} % Запрещаем переносы слов в оглавлении
%\setlength{\cftbeforechapterskip}{0pt} % Эта команда убирает интервал между заголовками глав - тут не надо, так красивее смотрится
\renewcommand{\aftertoctitle}{\afterchaptertitle \vspace{-\cftbeforechapterskip}} % Делаем отступ между словом "Содержание" и первой строкой таким же, как у заголовков глав
%\renewcommand*{\chapternumberline}[1]{} % Делаем так, чтобы номер главы не печатался - тут не надо
\renewcommand*{\cftchapternumwidth}{1.5em} % Ставим подходящий по размеру разделитель между номером главы и самим заголовком
\renewcommand*{\cftchapterfont}{\normalfont\MakeUppercase} % Названия глав обычным шрифтом заглавными буквами
\renewcommand*{\cftchapterpagefont}{\normalfont} % Номера страниц обычным шрифтом
\renewcommand*{\cftchapterdotsep}{\cftdotsep} % Делаем точки до номера страницы после названий глав
\renewcommand*{\cftdotsep}{1} % Задаем расстояние между точками
\renewcommand*{\cftchapterleader}{\cftdotfill{\cftchapterdotsep}} % Делаем точки стандартной формы (по умолчанию они "жирные")
\maxtocdepth{subsection} % В оглавление попадают только разделы первыхтрех уровней: главы, секции и подсекции

%%% Выравнивание и переносы %%%

%% http://tex.stackexchange.com/questions/241343/what-is-the-meaning-of-fussy-sloppy-emergencystretch-tolerance-hbadness
%% http://www.latex-community.org/forum/viewtopic.php?p=70342#p70342
\tolerance 1414
\hbadness 1414
\emergencystretch 1.5em                             % В случае проблем регулировать в первую очередь
\hfuzz 0.3pt
\vfuzz \hfuzz
%\dbottom
%\sloppy                                            % Избавляемся от переполнений
\clubpenalty=10000                                  % Запрещаем разрыв страницы после первой строки абзаца
\widowpenalty=10000                                 % Запрещаем разрыв страницы после последней строки абзаца
\brokenpenalty=4991                                 % Ограничение на разрыв страницы, если строка заканчивается переносом

%%% Объясняем компилятору, какие буквы русского алфавита можно использовать в перечислениях (подрисунках и нумерованных списках) %%%
%%% По ГОСТ нельзя использовать буквы ё, з, й, о, ч, ь, ы, ъ %%%
%%% Здесь также переопределены заглавные буквы, хотя в принципе они в документе не используются %%%

\makeatletter
    \def\russian@Alph#1{\ifcase#1\or
       А\or Б\or В\or Г\or Д\or Е\or Ж\or
       И\or К\or Л\or М\or Н\or
       П\or Р\or С\or Т\or У\or Ф\or Х\or
       Ц\or Ш\or Щ\or Э\or Ю\or Я\else\xpg@ill@value{#1}{russian@Alph}\fi}
    \def\russian@alph#1{\ifcase#1\or
       а\or б\or в\or г\or д\or е\or ж\or
       и\or к\or л\or м\or н\or
       п\or р\or с\or т\or у\or ф\or х\or
       ц\or ш\or щ\or э\or ю\or я\else\xpg@ill@value{#1}{russian@alph}\fi}
\makeatother

%%% Задаем параметры оформления рисунков и таблиц %%%

\usepackage{graphicx, caption, subcaption} % Подгружаем пакеты для работы с графикой и настройки подписей
\graphicspath{{images/}} % Определяем папку с рисунками
\captionsetup[figure]{font=small, width=\textwidth, name=Рисунок, justification=centering} % Задаем параметры подписей к рисункам: маленький шрифт (в данном случае 12pt), ширина равна ширине текста, полнотекстовая надпись "Рисунок", выравнивание по центру
\captionsetup[subfigure]{font=small} % Индексы подрисунков а), б) и так далее тоже шрифтом 12pt (по умолчанию делает еще меньше)
\captionsetup[table]{singlelinecheck=false,font=small,width=\textwidth,justification=justified} % Задаем параметры подписей к таблицам: запрещаем переносы, маленький шрифт (в данном случае 12pt), ширина равна ширине текста, выравнивание по ширине
\captiondelim{ --- } % Разделителем между номером рисунка/таблицы и текстом в подписи является длинное тире
\setkeys{Gin}{width=\textwidth} % По умолчанию размер всех добавляемых рисунков будет подгоняться под ширину текста
\renewcommand{\thesubfigure}{\asbuk{subfigure}} % Нумерация подрисунков строчными буквами кириллицы
%\setlength{\abovecaptionskip}{0pt} % Отбивка над подписью - тут не меняем
%\setlength{\belowcaptionskip}{0pt} % Отбивка под подписью - тут не меняем
\usepackage[section]{placeins} % Объекты типа float (рисунки/таблицы) не вылезают за границы секциии, в которой они объявлены

%%% Задаем параметры ссылок и гиперссылок %%% 

\usepackage{hyperref}                               % Подгружаем нужный пакет
\hypersetup{
    colorlinks=true,                                % Все ссылки и гиперссылки цветные
    linktoc=all,                                    % В оглавлении ссылки подключатся для всех отображаемых уровней
    linktocpage=true,                               % Ссылка - только номер страницы, а не весь заголовок (так выглядит аккуратнее)
    linkcolor=red,                                  % Цвет ссылок и гиперссылок - красный
    citecolor=red                                   % Цвет цитировний - красный
}

%%% Настраиваем отображение списков %%%

\usepackage{enumitem}                               % Подгружаем пакет для гибкой настройки списков
\renewcommand*{\labelitemi}{\normalfont{--}}        % В ненумерованных списках для пунктов используем короткое тире
\makeatletter
    \AddEnumerateCounter{\asbuk}{\russian@alph}     % Объясняем пакету enumitem, как использовать asbuk
\makeatother
\renewcommand{\labelenumii}{\asbuk{enumii})}        % Кириллица для второго уровня нумерации
\renewcommand{\labelenumiii}{\arabic{enumiii})}     % Арабские цифры для третьего уровня нумерации
\setlist{noitemsep, leftmargin=*}                   % Убираем интервалы между пунками одного уровня в списке
\setlist[1]{labelindent=\parindent}                 % Отступ у пунктов списка равен абзацному отступу
\setlist[2]{leftmargin=\parindent}                  % Плюс еще один такой же отступ для следующего уровня
\setlist[3]{leftmargin=\parindent}                  % И еще один для третьего уровня

%%% Счетчики для нумерации объектов %%%

\counterwithout{figure}{chapter}                    % Сквозная нумерация рисунков по документу
\counterwithout{equation}{chapter}                  % Сквозная нумерация математических выражений по документу
\counterwithout{table}{chapter}                     % Сквозная нумерация таблиц по документу

%%% Реализация библиографии пакетами biblatex и biblatex-gost с использованием движка biber %%%

\usepackage{csquotes} % Пакет для оформления сложных блоков цитирования (biblatex рекомендует его подключать)
\usepackage[%
backend=biber,                                      % Движок
bibencoding=utf8,                                   % Кодировка bib-файла
sorting=none,                                       % Настройка сортировки списка литературы
style=gost-numeric,                                 % Стиль цитирования и библиографии по ГОСТ
language=auto,                                      % Язык для каждой библиографической записи задается отдельно
autolang=other,                                     % Поддержка многоязычной библиографии
sortcites=true,                                     % Если в квадратных скобках несколько ссылок, то отображаться будут отсортированно
movenames=false,                                    % Не перемещать имена, они всегда в начале библиографической записи
maxnames=5,                                         % Максимальное отображаемое число авторов
minnames=3,                                         % До скольки сокращать число авторов, если их больше максимума
doi=false,                                          % Не отображать ссылки на DOI
isbn=false,                                         % Не показывать ISBN, ISSN, ISRN
]{biblatex}[2016/09/17]
\DeclareDelimFormat{bibinitdelim}{}                 % Убираем пробел между инициалами (Иванов И.И. вместо Иванов И. И.)
\addbibresource{bibl.bib}                           % Определяем файл с библиографией

%%% Скрипт, который автоматически подбирает язык (и, следовательно, формат) для каждой библиографической записи %%%
%%% Если в названии работы есть кириллица - меняем значение поля langid на russian %%%
%%% Все оставшиеся пустые места в поле langid заменяем на english %%%

\DeclareSourcemap{
  \maps[datatype=bibtex]{
    \map{
        \step[fieldsource=title, match=\regexp{^\P{Cyrillic}*\p{Cyrillic}.*}, final]
        \step[fieldset=langid, fieldvalue={russian}]
    }
    \map{
        \step[fieldset=langid, fieldvalue={english}]
    }
  }
}

%%% Прочие пакеты для расширения функционала %%%

\usepackage{longtable,ltcaption}                    % Длинные таблицы
\usepackage{multirow,makecell}                      % Улучшенное форматирование таблиц
\usepackage{booktabs}                               % Еще один пакет для красивых таблиц
\usepackage{soulutf8}                               % Поддержка переносоустойчивых подчёркиваний и зачёркиваний
\usepackage{icomma}                                 % Запятая в десятичных дробях
\usepackage{hyphenat}                               % Для красивых переносов
\usepackage{textcomp}                               % Поддержка "сложных" печатных символов типа значков иены, копирайта и т.д.
\usepackage[version=4]{mhchem}                      % Красивые химические уравнения
\usepackage{amsmath}                                % Усовершенствование отображения математических выражений 

%%% Вставляем по очереди все содержательные части документа %%%

\begin{document}

\thispagestyle{empty}

\begin{center}
    МИНИСТЕРСТВО ЦИФРОВОГО РАЗВИТИЯ, СВЯЗИ И МАССОВЫХ КОММУНИКАЦИЙ \\ РОССИЙСКОЙ ФЕДЕРАЦИИ

    \vspace{20pt}

    Федеральное государственное бюджетное образовательное учреждение  \\  высшего образования \\
    "<Сибирский государственный университет телекоммуникаций и информатики"> \\

    \vspace{20pt}

    Кафедра телекоммуникационных систем и вычислительных средств \\  (ТС и ВС)
\end{center}

\vfill

\begin{center}
    РЕФЕРАТ \\  
    по дисциплине \\
    \textit{"<Визуальное программирование">}

    \vspace{20pt}

    по теме: \\
    \uppercase{Создание MP3-плеера с помощью программы Android Studio}
\end{center}

\vfill

    \noindent Студент: \\
    \textit{Группа № ИА-331 \hfill И.А. Иванов}

    \vspace{20pt}

    \noindent Предподаватель: \\
    \textit{должность, уч. степень, уч. звание \hfill Р.В. Ахпашев}

\vfill

\begin{center}
    Новосибирск 2025 г.
\end{center}                                     % Титульник

\newpage % Переходим на новую страницу
\setcounter{page}{2} % Начинаем считать номера страниц со второй
\OnehalfSpacing* % Задаем полуторный интервал текста (в титульнике одинарный, поэтому команда стоит после него)

\tableofcontents*                                   % Автособираемое оглавление

\chapter*{Введение}
\addcontentsline{toc}{chapter}{Введение}
\label{ch:intro}

    В рамках данной работы необходимо создать приложение для воспроизведения аудио формата mp3.

    Для этого мы будем использовать компоненты для работы с плеером, представленные в программе \textbf{Android Studio}.

\endinput                                     % Введение
\chapter{Теория}
\label{ch:chap1}

    Поподробнее поговорим о том, каким образом будет устроен проект.  Приложение предоставляет следующие возможности:
    \begin{itemize}
        \item Воспроизведение аудиофайлов с устройства
        \item Управление плейлистами
        \item Отображение метаданных и обложек альбомов
        \item Система поиска и фильтрации треков
    \end{itemize}

\section{Переход в PlayerActivity}

    Переход осуществляется посредством обработки нажатия кнопки из MainActivity
    
\section{Работа с музыкой}

Для воспроизведения аудио используется класс \textbf{MediaPlayer}, который позволяет управлять воспроизведением, перемоткой и другими аспектами аудиофайлов. Для получения списка доступных песен применяется \textbf{MediaStore}, который предоставляет доступ к медиатеке устройства.

\subsection{Кнопки воспроизведения}

    В приложении есть 3 кнопки для работы с воспроизведением:

    \begin{itemize}
    \item Кнопка \textbf{"Play/Pause"} - служит для воспроизведения/паузы трека
    \item Кнопка \textbf{"Next"} - переключает текущий трек на следующий в очереди
    \item Кнопка \textbf{"Previous"} - необходима для воспроизведения предыдущего в очереди трека
    \item Ползунок - с его помощью можно перематывать трек на необходимую длину
    \end{itemize}        

\subsection{Кнопки для работы с музыкой и плейлистами}

    \begin{itemize}
    \item Кнопка \textbf{"Выбрать музыку"} - позволяет выбрать трек, содержащийся на телефоне
    \item Кнопка \textbf{"Все песни"} - показывает все треки, из выбранных пользователем
    \item Кнопка \textbf{"Создать плейлист"} - с ее помощью создается плейлист, в котором вводится название и выбираютяя треки, которые в него войдут
    \item Кнопка \textbf{"Мои плейлисты"} - позволяет выбрать плейлист из созданных
    \end{itemize}
          
\section{Сохранение данных}


Для хранения информации о созданных плейлистах используется механизм сериализации данных в JSON-формат с помощью библиотеки Gson и сохранение их в \textbf{SharedPreferences} — легкий механизм хранения настроек и небольших данных.

\section{Создание плейлиста}

    При нажатии на кнопку "создать плейлист" пользователь выбирает, как будет называться плейлист и какие треки будут в него входить

    Также пользователь имеет возможность выбирать, какой именно плейлист он хочет открыть.

\endinput                                     % Первая глава
\chapter{Практика}
\label{ch:chap2}

Далле будет представлена программная реализации описанного ранее

\section{Основной класс: Player}

Это AppCompatActivity, главный экран приложения. Он содержит всю логику работы плеера.

	
\subsection{Основные переменные и компоненты UI}

MediaPlayer — для воспроизведения аудио:

SeekBar — ползунок для отображения прогресса текущего трека и его управления:

Кнопки: Play/Pause (playButton), Next (nextButton), Previous (prevButton), а также кнопки для выбора музыки, просмотра всех песен, создания плейлистов и просмотра существующих.

TextView: trackName — название текущего трека.

ImageView: albumArt — изображение обложки альбома.

SharedPreferences — для сохранения данных о плейлистах.


\textbf{Программная реализация:}
\begin{verbatim}
	private lateinit var mediaPlayer: MediaPlayer
    private lateinit var seekBar: SeekBar
    private lateinit var playButton: Button
    private lateinit var nextButton: Button
    private lateinit var prevButton: Button
    private lateinit var trackName: TextView
    private lateinit var selectMusicButton: Button
    private lateinit var showSongsButton: Button
    private lateinit var createPlaylistButton: Button
    private lateinit var showPlaylistsButton: Button
    private lateinit var albumArt: ImageView

    private val handler = Handler(Looper.getMainLooper())
    private var isPlaying = false
    private var currentTrackIndex = 0
    private var tracks = mutableListOf<Track>()
    private var playlists = mutableListOf<Playlist>()
    private lateinit var sharedPrefs: SharedPreferences
\end{verbatim}

\subsection{Данные}

Track — класс данных для хранения информации о песне: URI файла, название, ID альбома и длительность.

Playlist — класс данных для хранения названия плейлиста и списка треков.


\textbf{Программная реализация:}
\begin{verbatim}
	data class Track(
        val uri: Uri,
        val name: String,
        val albumId: Long,
        val duration: Long
    )

    data class Playlist(
        val name: String,
        val tracks: MutableList<Track> = mutableListOf()
    )
\end{verbatim}

\section{Инициализация (onCreate)}

Устанавливается layout.

Инициализируются UI-компоненты (initViews()).

Настраивается MediaPlayer.

Настраиваются обработчики кнопок (setupButtons()).

Загружаются сохранённые плейлисты из SharedPreferences.

Проверяется разрешение на чтение внешнего хранилища (checkPermissions()).


\textbf{Программная реализация:}
\begin{verbatim}
	override fun onCreate(savedInstanceState: Bundle?) {
        super.onCreate(savedInstanceState)
        setContentView(R.layout.activity_player2)

        initViews()
        setupMediaPlayer()
        setupButtons()
        loadPlaylists()
        checkPermissions()
    }

    private fun initViews() {
        seekBar = findViewById(R.id.seekBar)
        playButton = findViewById(R.id.playButton)
        nextButton = findViewById(R.id.nextButton)
        prevButton = findViewById(R.id.prevButton)
        trackName = findViewById(R.id.trackName)
        selectMusicButton = findViewById(R.id.selectMusicButton)
        showSongsButton = findViewById(R.id.showSongsButton)
        createPlaylistButton = findViewById(R.id.createPlaylistButton)
        showPlaylistsButton = findViewById(R.id.showPlaylistsButton)
        albumArt = findViewById(R.id.albumArt)
        sharedPrefs = getSharedPreferences(PREFS_NAME, MODE_PRIVATE)
        albumArt.setImageResource(R.drawable.ic_music_note)
    }

    private fun setupMediaPlayer() {
        mediaPlayer = MediaPlayer().apply {
            setOnCompletionListener { nextTrack() }
            setOnPreparedListener {
                start()
                this@Player.isPlaying = true
                playButton.text = "Pause"
                startSeekbarUpdate()
            }
            setOnErrorListener { _, what, extra ->
                Log.e(TAG, "Error what=$what extra=$extra")
                showToast("Ошибка воспроизведения")
                false
            }
        }
    }

    private fun setupButtons() {
        selectMusicButton.setOnClickListener { checkPermissionAndBrowseMusic() }
        showSongsButton.setOnClickListener { showAllSongsDialog() }
        createPlaylistButton.setOnClickListener { showCreatePlaylistDialog() }
        showPlaylistsButton.setOnClickListener { showPlaylistsDialog() }

        playButton.setOnClickListener {
            if (tracks.isEmpty()) {
                showToast("Сначала выберите музыку")
                return@setOnClickListener
            }
            if (isPlaying) pauseMusic() else playMusic()
        }

        nextButton.setOnClickListener {
            if (tracks.isEmpty()) {
                showToast("Сначала выберите музыку")
                return@setOnClickListener
            }
            nextTrack()
        }

        prevButton.setOnClickListener {
            if (tracks.isEmpty()) {
                showToast("Сначала выберите музыку")
                return@setOnClickListener
            }
            previousTrack()
        }

        seekBar.setOnSeekBarChangeListener(object : SeekBar.OnSeekBarChangeListener {
            override fun onProgressChanged(seekBar: SeekBar?, progress: Int, fromUser: Boolean) {
                if (fromUser && mediaPlayer.isPlaying) {
                    mediaPlayer.seekTo(progress)
                }
            }
            override fun onStartTrackingTouch(seekBar: SeekBar?) {}
            override fun onStopTrackingTouch(seekBar: SeekBar?) {}
        })
    }

    private fun checkPermissions() {
        if (ContextCompat.checkSelfPermission(
                this,
                Manifest.permission.READ_EXTERNAL_STORAGE
            ) != PackageManager.PERMISSION_GRANTED
        ) {
            ActivityCompat.requestPermissions(
                this,
                arrayOf(Manifest.permission.READ_EXTERNAL_STORAGE),
                REQUEST_PERMISSION
            )
        }
    }
\end{verbatim}

\section{Работа с разрешениями}

Если разрешение не предоставлено, запрашивается у пользователя. После получения разрешения вызывается loadTracks(), которая загружает все песни из внешнего хранилища.

\textbf{Программная реализация:}
\begin{verbatim}
	override fun onRequestPermissionsResult(
        requestCode: Int,
        permissions: Array<out String>,
        grantResults: IntArray
    ) {
        super.onRequestPermissionsResult(requestCode, permissions, grantResults)
        when (requestCode) {
            REQUEST_PERMISSION -> {
                if (grantResults.isNotEmpty() && grantResults[0] == PackageManager.PERMISSION_GRANTED) {
                    loadTracks()
                } else {
                    showToast("Необходимы разрешения для доступа к музыке")
                }
            }
        }
    }

    private fun loadTracks() {
        tracks.addAll(getAllSongsFromStorage())
        if (tracks.isNotEmpty()) {
            playTrack(0)
        }
    }
\end{verbatim}

\section{Получение песен из хранилища}

Метод getAllSongsFromStorage.Он собирает список объектов Track.

Также есть метод addTrackFromUri(), который добавляет трек по URI, полученному при выборе файла через диалог.

\textbf{Программная реализация:}
\begin{verbatim}
	private fun getAllSongsFromStorage(): List<Track> {
        val songs = mutableListOf<Track>()
        if (ContextCompat.checkSelfPermission(
                this,
                Manifest.permission.READ_EXTERNAL_STORAGE
            ) != PackageManager.PERMISSION_GRANTED
        ) {
            return songs
        }

        val projection = arrayOf(
            MediaStore.Audio.Media._ID,
            MediaStore.Audio.Media.TITLE,
            MediaStore.Audio.Media.ALBUM_ID,
            MediaStore.Audio.Media.DURATION
        )

        contentResolver.query(
            MediaStore.Audio.Media.EXTERNAL_CONTENT_URI,
            projection,
            "${MediaStore.Audio.Media.IS_MUSIC} != 0",
            null,
            "${MediaStore.Audio.Media.TITLE} ASC"
        )?.use { cursor ->
            while (cursor.moveToNext()) {
                val id = cursor.getLong(cursor.getColumnIndexOrThrow(MediaStore.Audio.Media._ID))
                val name = cursor.getString(cursor.getColumnIndexOrThrow(MediaStore.Audio.Media.TITLE))
                val albumId = cursor.getLong(cursor.getColumnIndexOrThrow(MediaStore.Audio.Media.ALBUM_ID))
                val duration = cursor.getLong(cursor.getColumnIndexOrThrow(MediaStore.Audio.Media.DURATION))
                val uri = ContentUris.withAppendedId(MediaStore.Audio.Media.EXTERNAL_CONTENT_URI, id)
                songs.add(Track(uri, name, albumId, duration))
            }
        }
        return songs
    }

	private fun addTrackFromUri(uri: Uri) {
        try {
            val projection = arrayOf(
                MediaStore.Audio.Media.TITLE,
                MediaStore.Audio.Media.ALBUM_ID,
                MediaStore.Audio.Media.DURATION
            )

            contentResolver.query(uri, projection, null, null, null)?.use { cursor ->
                if (cursor.moveToFirst()) {
                    val name = cursor.getString(cursor.getColumnIndexOrThrow(MediaStore.Audio.Media.TITLE))
                    val albumId = cursor.getLong(cursor.getColumnIndexOrThrow(MediaStore.Audio.Media.ALBUM_ID))
                    val duration = cursor.getLong(cursor.getColumnIndexOrThrow(MediaStore.Audio.Media.DURATION))
                    tracks.add(Track(uri, name, albumId, duration))
                }
            }
        } catch (e: Exception) {
            Log.e(TAG, "Error adding track from URI", e)
        }
    }
\end{verbatim}

\section{Воспроизведение музыки}

Основной метод — playTrack(index). Он:

Сбрасывает текущий MediaPlayer.

Устанавливает источник данных (URI выбранного трека).

Готовит плеер асинхронно (prepareAsync()).

Обновляет название трека и изображение обложки.

После подготовки запускает воспроизведение (start()).

Обложка альбома загружается через ContentUri с ID альбома.

\textbf{Программная реализация:}
\begin{verbatim}
	private fun playTrack(index: Int) {
        if (index < 0 || index >= tracks.size) return

        try {
            mediaPlayer.reset()
            mediaPlayer.setDataSource(applicationContext, tracks[index].uri)
            mediaPlayer.prepareAsync()
            currentTrackIndex = index
            trackName.text = tracks[index].name
            loadAlbumArt(tracks[index].albumId)
        } catch (e: Exception) {
            Log.e(TAG, "Play track error", e)
            showToast("Ошибка: ${e.message}")
        }
    }
\end{verbatim}

\section{РУправление воспроизведением}

Кнопки:

Play/Pause: запускает или ставит на паузу текущий трек.

Next/Previous: переключают на следующий или предыдущий трек в списке.

Автоматически при завершении трека вызывается nextTrack(), чтобы перейти к следующему.

\textbf{Программная реализация:}
\begin{verbatim}
private fun nextTrack() {
        if (tracks.isEmpty()) return
        currentTrackIndex = if (currentTrackIndex < tracks.size - 1) currentTrackIndex + 1 else 0
        playTrack(currentTrackIndex)
    }
\end{verbatim}

\section{Обновление прогресс-бара}

Метод startSeekbarUpdate() запускает цикл обновления прогресса каждые 1 секунду через Handler, синхронизируя SeekBar с текущим положением воспроизведения.

Пользователь может перемещать ползунок для перемотки трека.

\textbf{Программная реализация:}
\begin{verbatim}
	private fun startSeekbarUpdate() {
        handler.post(object : Runnable {
            override fun run() {
                if (mediaPlayer.isPlaying) {
                    seekBar.progress = mediaPlayer.currentPosition
                    seekBar.max = mediaPlayer.duration
                    handler.postDelayed(this, 1000)
                }
            }
        })
    }
\end{verbatim}

\section{Работа с диалогами}

\subsection{Все песни (showAllSongsDialog())}

Показывает список всех песен в виде RecyclerView с адаптером SongAdapter. При выборе песни она добавляется в текущий список воспроизведения и запускается.

\textbf{Программная реализация:}
\begin{verbatim}
	private fun showAllSongsDialog() {
        val dialogView = LayoutInflater.from(this).inflate(R.layout.dialog_song_list, null)
        val recyclerView = dialogView.findViewById<RecyclerView>(R.id.songsRecyclerView)
        val adapter = SongAdapter(getAllSongsFromStorage()) { track ->
            tracks.clear()
            tracks.add(track)
            currentTrackIndex = 0
            playTrack(currentTrackIndex)
        }

        recyclerView.layoutManager = LinearLayoutManager(this)
        recyclerView.adapter = adapter

        AlertDialog.Builder(this)
            .setTitle("Все песни")
            .setView(dialogView)
            .setPositiveButton("Закрыть", null)
            .show()
    }
\end{verbatim}

\subsection{Создание плейлиста (showCreatePlaylistDialog())}

Позволяет выбрать несколько песен (режим мультивыбора) и задать название плейлиста. После подтверждения создаётся объект Playlist, он сохраняется в списке и сериализуется в SharedPreferences.

\textbf{Программная реализация:}
\begin{verbatim}
	private fun showCreatePlaylistDialog() {
        val dialogView = LayoutInflater.from(this).inflate(R.layout.dialog_create_playlist, null)
        val playlistNameEditText = dialogView.findViewById<EditText>(R.id.playlistNameEditText)
        val songsRecyclerView = dialogView.findViewById<RecyclerView>(R.id.songsRecyclerView)

        val allSongs = getAllSongsFromStorage()
        val adapter = SongAdapter(allSongs) { }
        adapter.setMultiSelectMode(true)

        songsRecyclerView.layoutManager = LinearLayoutManager(this)
        songsRecyclerView.adapter = adapter

        AlertDialog.Builder(this)
            .setTitle("Создать плейлист")
            .setView(dialogView)
            .setPositiveButton("Создать") { _, _ ->
                val playlistName = playlistNameEditText.text.toString()
                if (playlistName.isNotEmpty() && adapter.getSelectedSongs().isNotEmpty()) {
                    val newPlaylist = Playlist(playlistName)
                    newPlaylist.tracks.addAll(adapter.getSelectedSongs())
                    playlists.add(newPlaylist)
                    savePlaylists()
                    showToast("Плейлист создан")
                } else {
                    showToast("Введите название и выберите песни")
                }
            }
            .setNegativeButton("Отмена", null)
            .show()
    }
\end{verbatim}

\subsection{Просмотр плейлистов (showPlaylistsDialog())}

Показывает список созданных плейлистов. При клике загружается их содержимое в текущий список воспроизведения и запускается первая песня.

Также реализовано удаление плейлиста по долгому нажатию с подтверждением.

\textbf{Программная реализация:}
\begin{verbatim}
	private fun showPlaylistsDialog() {
        if (playlists.isEmpty()) {
            showToast("У вас пока нет плейлистов")
            return
        }

        val dialogView = LayoutInflater.from(this).inflate(R.layout.dialog_playlist_list, null)
        val recyclerView = dialogView.findViewById<RecyclerView>(R.id.playlistsRecyclerView)

        val adapter = object : RecyclerView.Adapter<PlaylistViewHolder>() {
            override fun onCreateViewHolder(parent: ViewGroup, viewType: Int): PlaylistViewHolder {
                val view = LayoutInflater.from(parent.context)
                    .inflate(R.layout.item_playlist, parent, false)
                return PlaylistViewHolder(view, this)
            }

            override fun onBindViewHolder(holder: PlaylistViewHolder, position: Int) {
                val playlist = playlists[position]
                holder.bind(playlist) {
                    tracks.clear()
                    tracks.addAll(playlist.tracks)
                    currentTrackIndex = 0
                    playTrack(currentTrackIndex)
                }
            }

            override fun getItemCount(): Int = playlists.size
        }

        recyclerView.layoutManager = LinearLayoutManager(this)
        recyclerView.adapter = adapter

        AlertDialog.Builder(this)
            .setTitle("Мои плейлисты")
            .setView(dialogView)
            .setPositiveButton("Закрыть", null)
            .show()
    }
\end{verbatim}

\section{Сохранение и загрузка плейлистов}


Используется Gson для сериализации/десериализации списка плейлистов в JSON строку, которая хранится в SharedPreferences под ключом "playlists".

\section{Дополнительные компоненты}

\subsection{Адаптеры RecyclerView}

SongAdapter — отображает список песен с возможностью мультивыбора.

PlaylistViewHolder внутри метода отображения плейлистов — отображает название, количество треков и изображение обложки; поддерживает удаление по долгому нажатию.

\textbf{Программная реализация:}
\begin{verbatim}

	inner class PlaylistViewHolder(
        itemView: View,
        private val adapter: RecyclerView.Adapter<*>
    ) : RecyclerView.ViewHolder(itemView) {
        private val playlistName: TextView = itemView.findViewById(R.id.playlistName)
        private val trackCount: TextView = itemView.findViewById(R.id.trackCount)
        private val playlistArt: ImageView = itemView.findViewById(R.id.playlistArt)

        fun bind(playlist: Playlist, onClick: () -> Unit) {
            playlistName.text = playlist.name
            trackCount.text = "${playlist.tracks.size} треков"

            if (playlist.tracks.isNotEmpty()) {
                val albumArtUri = ContentUris.withAppendedId(
                    Uri.parse("content://media/external/audio/albumart"),
                    playlist.tracks[0].albumId
                )
                playlistArt.setImageURI(albumArtUri)
                if (playlistArt.drawable == null) {
                    playlistArt.setImageResource(R.drawable.ic_music_note)
                }
            } else {
                playlistArt.setImageResource(R.drawable.ic_music_note)
            }

            itemView.setOnClickListener { onClick() }

            itemView.setOnLongClickListener {
                AlertDialog.Builder(itemView.context)
                    .setTitle("Удалить плейлист?")
                    .setMessage("Вы уверены, что хотите удалить '${playlist.name}'?")
                    .setPositiveButton("Удалить") { _, _ ->
                        playlists.removeAt(adapterPosition)
                        savePlaylists()
                        adapter.notifyItemRemoved(adapterPosition)
                        showToast("Плейлист удален")
                    }
                    .setNegativeButton("Отмена", null)
                    .show()
                true
            }
        }
    }

class SongAdapter(
    private val songs: List<Player.Track>,
    private val onItemClick: (Player.Track) -> Unit
) : RecyclerView.Adapter<SongAdapter.SongViewHolder>() {

    private val selectedSongs = mutableSetOf<Player.Track>()
    private var isMultiSelectMode = false

    fun setMultiSelectMode(enabled: Boolean) {
        isMultiSelectMode = enabled
        notifyDataSetChanged()
    }

    fun getSelectedSongs(): List<Player.Track> {
        return selectedSongs.toList()
    }

    inner class SongViewHolder(itemView: View) : RecyclerView.ViewHolder(itemView) {
        private val songName: TextView = itemView.findViewById(R.id.songName)
        private val songDuration: TextView = itemView.findViewById(R.id.songDuration)
        private val albumArt: ImageView = itemView.findViewById(R.id.albumArt)

        fun bind(track: Player.Track) {
            songName.text = track.name
            songDuration.text = formatDuration(track.duration)

            val albumArtUri = ContentUris.withAppendedId(
                Uri.parse("content://media/external/audio/albumart"),
                track.albumId
            )
            albumArt.setImageURI(albumArtUri)
            if (albumArt.drawable == null) {
                albumArt.setImageResource(R.drawable.ic_music_note)
            }

            itemView.isSelected = selectedSongs.contains(track)
            itemView.setOnClickListener {
                if (isMultiSelectMode) {
                    toggleSelection(track)
                } else {
                    onItemClick(track)
                }
            }
        }

        private fun toggleSelection(track: Player.Track) {
            if (selectedSongs.contains(track)) {
                selectedSongs.remove(track)
            } else {
                selectedSongs.add(track)
            }
            notifyItemChanged(adapterPosition)
        }

        private fun formatDuration(duration: Long): String {
            val seconds = (duration / 1000) % 60
            val minutes = (duration / (1000 * 60)) % 60
            return String.format("%02d:%02d", minutes, seconds)
        }
    }

    override fun onCreateViewHolder(parent: ViewGroup, viewType: Int): SongViewHolder {
        val view = LayoutInflater.from(parent.context)
            .inflate(R.layout.item_song, parent, false)
        return SongViewHolder(view)
    }

    override fun onBindViewHolder(holder: SongViewHolder, position: Int) {
        holder.bind(songs[position])
    }

    override fun getItemCount(): Int = songs.size
}
\end{verbatim}



\endinput                                     % Вторая глава
\chapter{Заключение}

В ходе выполненной работы была создана полноценная мобильная музыкальная приложение для платформы Android на языке Kotlin. Реализован функционал, включающий загрузку и отображение музыкальных файлов из внешнего хранилища, управление воспроизведением (воспроизведение, пауза, переключение треков), отображение информации о текущем треке и его обложки.

Дополнительно реализованы возможности создания пользовательских плейлистов, их сохранения и последующего просмотра и воспроизведения. В приложении использованы современные компоненты Android, такие как RecyclerView для отображения списков, а также реализована работа с разрешениями и сохранением данных через SharedPreferences с использованием Gson.

Работа продемонстрировала навыки интеграции мультимедийных API, работы с пользовательским интерфейсом и хранения данных, что является важным этапом в разработке мобильных приложений. Итоговая реализация обеспечивает удобство использования и расширяемость, что создает хорошую базу для дальнейшего развития проекта или внедрения дополнительных функций.

\endinput                                     % Третья глава
\chapter{Список использованных источников}

Документация Kotlin: https://kotlinlang.org/docs/home.html

Документация Android Jetpack: https://developer.android.com/jetpack

\endinput                                     % Четвертая глава

\printbibliography[title=Список использованных источников] % Автособираемый список литературы

\end{document}