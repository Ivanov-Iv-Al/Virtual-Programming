\chapter{Теория}
\label{ch:chap1}

    Поподробнее поговорим о том, каким образом будет устроен проект.  Приложение предоставляет следующие возможности:
    \begin{itemize}
        \item Воспроизведение аудиофайлов с устройства
        \item Управление плейлистами
        \item Отображение метаданных и обложек альбомов
        \item Система поиска и фильтрации треков
    \end{itemize}

\section{Переход в PlayerActivity}

    Переход осуществляется посредством обработки нажатия кнопки из MainActivity
    
\section{Работа с музыкой}

Для воспроизведения аудио используется класс \textbf{MediaPlayer}, который позволяет управлять воспроизведением, перемоткой и другими аспектами аудиофайлов. Для получения списка доступных песен применяется \textbf{MediaStore}, который предоставляет доступ к медиатеке устройства.

\subsection{Кнопки воспроизведения}

    В приложении есть 3 кнопки для работы с воспроизведением:

    \begin{itemize}
    \item Кнопка \textbf{"Play/Pause"} - служит для воспроизведения/паузы трека
    \item Кнопка \textbf{"Next"} - переключает текущий трек на следующий в очереди
    \item Кнопка \textbf{"Previous"} - необходима для воспроизведения предыдущего в очереди трека
    \item Ползунок - с его помощью можно перематывать трек на необходимую длину
    \end{itemize}        

\subsection{Кнопки для работы с музыкой и плейлистами}

    \begin{itemize}
    \item Кнопка \textbf{"Выбрать музыку"} - позволяет выбрать трек, содержащийся на телефоне
    \item Кнопка \textbf{"Все песни"} - показывает все треки, из выбранных пользователем
    \item Кнопка \textbf{"Создать плейлист"} - с ее помощью создается плейлист, в котором вводится название и выбираютяя треки, которые в него войдут
    \item Кнопка \textbf{"Мои плейлисты"} - позволяет выбрать плейлист из созданных
    \end{itemize}
          
\section{Сохранение данных}


Для хранения информации о созданных плейлистах используется механизм сериализации данных в JSON-формат с помощью библиотеки Gson и сохранение их в \textbf{SharedPreferences} — легкий механизм хранения настроек и небольших данных.

\section{Создание плейлиста}

    При нажатии на кнопку "создать плейлист" пользователь выбирает, как будет называться плейлист и какие треки будут в него входить

    Также пользователь имеет возможность выбирать, какой именно плейлист он хочет открыть.

\endinput